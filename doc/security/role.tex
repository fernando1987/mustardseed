\section{基于角色的Web安全模块}
该模块利用Java的反射机制,对来自Web的请求,在Controller层做基于角色的
访问控制。

\subsection{Annotation}
角色控制的Annotation在org.mustardseed.security.annotation包中。
\begin{verbatim}
public @interface RoleAuthStamp {
    public String[] value() default {};
    public String[] except() default {};
    public String error() default "";
}
\end{verbatim}

\begin{description}
\item[value] 由角色字符串构成的一个数组,在数组中的角色将被认为合法。
\item[except] 由角色字符串构成的一个数组,在数组中的角色将被认为非法。
\item[error] 一个字符串类型的自定义错误代码。
\end{description}

自定义错误代码在SprintMVC 与 Struts2中的意义是不同的。在SpringMVC
中,错误代码将被添加至AuthResult中,而在Struts2中,将被作为请求为
非法请求之后,所跳转的Result名。

\subsection{安全模块}
角色控制的安全模块位于org.mustardseed.security包中的 WebCtrlRoleResolver
若要让其正常使用只需在Spring中用IoC设置roleAuthBean 和 defaultErrorCode 
两个参数即可。
\begin{description}
\item[roleAuthBean]是一个实现 RoleAuthenticationBean 接口的用户自定义对象,
用于完成用户访问合法性的判断。
\item[defaultErrorCode]是一个字符串类型,当在
类与成员函数上都未设置error时,作为默认error返回值使用。
\end{description}

\subsection{角色访问控制实现}
RoleAuthenticationBean 是用户自定义实现的一个接口。用户只需要实现 getAuthRoles
 和 checkAuth 两个成员方法即可。
\begin{description}
\item[getAuthRoles] 返回所有有效的角色字符串集合。即角色集合的全集。
\item[checkAuth] 接受一个当前请求的合法角色集合,与NativeWebRequest 对象。
用户可在其中根据合法用户集合,与从NativeWebRequest中获得的session信息等,
来判断该请求是否合法。若合法则回返``true'',不然返回``false''
\end{description}


