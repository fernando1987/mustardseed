\section{访问来源控制}
该模块利用Java的反射机制,判断其Web请求的来源是否合法。可用于低强度的
防盗链,跨站攻击等。

\subsection{Annotation}
来源控制的Annotation在org.mustardseed.security.annotation包中。
\begin{verbatim}
public @interface WebRefStamp {
    public String[] value() default {};
    public boolean passWhenNull() default true;
    public RequestMethod[] method() 
        default {RequestMethod.GET, RequestMethod.POST};
    public String error() default "";
}
\end{verbatim}

\begin{description}
\item[value] 允许的来源url列表名,可选择多个列表。
\item[passWhenNull] 设置是否允许未知来源的链接请求。
\item[RequestMethod] 允许的请求方法。
\item[error] 自定义错误代码。
\end{description}

\subsection{安全模块}
来源控制的安全模块位于org.mustardseed.security包中的 WebCtrlRefResolver
在使用时只需在Spring中用IoC设置如下的几个参数即可。
\begin{description}
\item[urlDict] 这个参数为一个map对象,定义不同名字的URL列表。在列表中的URL
将被用于判断链接来源的合法性。
\item[isMatchProtocol] 设置是否判断URL协议。如http、ftp等。(可选)
\item[isMatchHost] 设置是否判断主机地址。(可选)
\item[isMatchPort] 设置是否判断URL中的端口号。(可选)
\end{description}

在设置中,默认情况下三个可选项都为``false'',通常设置isMatchHost为``true''即可。
若都为``false''则对于URL限制的访问控制将失效。在实际使用中urlDict可以设置多个
不同名的列表,针对不同的场合,采用不同的列表进行URL访问控制。
