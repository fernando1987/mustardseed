
\section{Validation概览与介绍}
Validation是一个基于Web表单的轻量级表单验证机制。首先,验证是基于表单规则的。其次,
统一的意义在于,客户端验证和服务器端验证的实现只要实现一次即可。最后,该机制是一个
轻量级实现。该机制的目标是实现一个几乎零学习成本,且便于单元测试,相对灵活的,表单
验证机制。

\subsection{特性介绍}
对于一般的表单验证,为了减轻服务器端压力,会在客户端处先做一个表单验证。等到数据递交
之后,再在服务器端做一次几乎相同的表单验证。但是,由于服务器端实现和客户端实现是不同
的。因此,一般的实现是在客户端实现套表单验证,再在服务器端再实现一套表单验证。但是,
该本机制的实现在保证了表单验证的相对灵活性和可扩展性的同时,让表单验证的代码只需要
一次,即可同时在客户端和服务器端实现验证。

\subsection{框架简介}
框架是基于Spring进行配置的,配置完成后,只需在action层使用注解标注,即可完成对表单
的验证配置。客户端采用的是jquery,但不限于特定框架,只要实现相关接口的封装调用即可。
用户在递交完表单之后,会通过客户端的javascript脚本在客户端处据根据你所编写的验证
脚本进行数据验证。表单递交后,在服务器端会采用相同的验证脚本进行同样的验证。两次验证
都会返回错误信息,使用者只要对错误信息进行处理即可。
