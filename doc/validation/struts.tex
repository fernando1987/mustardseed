\section{Struts下的配置和使用}
在 Struts2 下,采用的是拦截器实现。拦截器将根据脚本进行数据校验,并按照Struts的默认
表单校验方式处理错误信息,即将错误信息内容丢给实现了 ValidationAware 接口的 Action
然后返回 INPUT 的 Result 。之后在服务器端的表单验证处理逻辑就和默认的验证逻辑基本相同。

\subsection{拦截器配置}
拦截器的配置分成两步,一个是配置Struts2的内容拦截器,另一个是用于实际处理表单校验的
SpringBean。

以下是Struts2拦截器的配置。
\begin{verbatim}
<interceptors>
  <interceptor name="validationInterceptor"
    class="org.mustardseed.validation.struts.ValidationHandler"/>
  <interceptor-stack name="mustardDefaultStack">
    <interceptor-ref name="validationInterceptor"/>
    <interceptor-ref name="defaultStack"/>
  </interceptor-stack>
</interceptors>
<default-interceptor-ref name="mustardDefaultStack"/>
\end{verbatim}

以下是Spring相关的配置,其中 validator.js 是你存放所有校验函数的js脚本的服务器路径。
\begin{verbatim}
<bean name="validator"
      class="org.mustardseed.validation.Validator">
  <property name="scriptPath" value="validator.js"/>
</bean>
\end{verbatim}

\subsection{Validation的使用}

在Struts2中使用的时候,只需要在你响应的method上,加上一个注解即可。下面是实例,
其中的 userRegisterValid 为你页面上用来校验表单的校验函数名。请确保两个名字
相同,不然校验结果将会不一致。
\begin{verbatim}
@FormValidResult("userRegisterValid")
public String execute() {
  return SUCCESS;
}
\end{verbatim}
